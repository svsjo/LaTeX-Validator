% Abkürzungen
\newacronym{arburg}{ARBURG}{ARBURG GmbH \& Co KG}

\newacronym{gestica}{GESTICA}{GESTICA Maschinensteuerung}

\newacronym{selogica}{SELOGICA}{SELOGICA Maschinensteuerung}

\newacronym{cca}{CCA}{Composite Components Architecture}

\newacronym{coco}{CoCo}{Composite Components}

\newacronym{mvvm}{MVVM}{Model-View-ViewModel}

\newacronym{xaml}{XAML}{Extensible Application Markup Language}

\newacronym{wpf}{WPF}{Windows Presentation Foundation}

\newacronym{ui}{UI}{User Interface}

\newacronym{api}{API}{Application Programming Interface}

\newacronym{clr}{CLR}{Common Language Runtime}

\newacronym{sdk}{SDK}{Software Development Kit}

\newacronym{oop}{OOP}{Objektorientierte Programmierung}

\newacronym{linq}{LINQ}{Language Integrated Query}

\newacronym{di}{DI}{Dependency Injection}


%%%%%%%%%%%
% Glossar % %%%%%%%%%%%%%%%%%%%%%
%%%%%%%%%%%
\newglossaryentry{freeformer}{name={Freeformer},description={Maschine zur additiven Fertigung von Kunststoffbauteilen}}

\newglossaryentry{allrounder}{name={ALLROUNDER},description={Spritzgießmaschine zur Kunststoffverarbeitung}}

\newglossaryentry{framework}{name={Framework},description={Programmiergerüst der Softwareentwicklung}}

\newglossaryentry{opensource}{name={Open Source},description={Software, deren Quelltext eingesehen, genutzt und verändert werden darf}}

\newglossaryentry{dotnet_framework}{name={.NET},description={Das vereinheitlichte Framework zur Entwicklung auf Microsoft-Betriebssystemen}}

\newglossaryentry{wrapper}{name={Wrapper},description={Ein Adapter, welcher eine Software umgibt um Kompatibilitäts-, Sicherheits- oder architiktonische Probleme zu lösen}}

\newglossaryentry{class_diagram}{name={Klassendiagramm},description={Statisches Diagramm zur Darstellung von Klassen, Schnittstellen und Beziehungen}}

\newglossaryentry{sequence_diagram}{name={Sequenzdiagramm},description={Interaktions-Diagramm zur zeitlich korrekten Darstellung von Abläufen zwischen Objekten}}

\newglossaryentry{interface}{name={Interface},description={Schnittstelle zur Definition von Methoden und Attributen, welche in einer Klasse vorhanden sein müssen}}

\newglossaryentry{getter}{name={Getter},description={Zugriffsfunktion zur Abfrage des Wertes eines Attributes}}

% Datei erzeugen
\makeglossaries